\documentclass[10pt,twocolumn,twoside,openany]{book}
\usepackage{dnd}
\usepackage{fullpage}
\usepackage[utf8]{inputenc}
\usepackage[letterpaper]{geometry}
\newgeometry{right=0.6in, top=0.75in, bottom=0.75in}
\usepackage[english]{babel}
\usepackage{indentfirst}
\usepackage{titlesec}
\usepackage{fancyhdr}
\usepackage[T1]{fontenc}
\usepackage[pdftex]{hyperref}
\usepackage{bookmark}
\usepackage{makecell}
\usepackage{graphicx}
\usepackage{easyfig}
/$ if variables.coverimage $/
\usepackage[pages=some]{background}
\backgroundsetup{
scale=1,
color=black,
opacity=1,
angle=0,
contents={%
  \includegraphics[width=\paperwidth,height=\paperheight]{/@variables.coverimage@/}
  }%
}
/$ endif $/
\usepackage[absolute,overlay]{textpos}
\pdfadjustspacing=1
\pdfcompresslevel=9
\DeclareGraphicsExtensions{.pdf,.png,.jpg,.jpeg}

\usepackage[strict]{changepage}

\hypersetup{
	pdfcreator={pdflatex},
	pdfauthor={/@ variables.author @/},
    pdftitle={/@ variables.title @/},
    pdfsubject={Dungeons and Dragons},
    pdfkeywords={Dungeons and Dragons, D\&D, 5e, adventure},
    bookmarks=true,
    linkcolor=blue,
    urlcolor=blue,
    citecolor=black,
    colorlinks=true,
}

\fancypagestyle{dndstyle}{
	\fancyhf{}
	\fancyfoot[CE,CO]{\fontfamily{\rmdefault}\selectfont /@ variables.title @/}
	\renewcommand{\headrulewidth}{0pt}
	\fancyfoot[LE,RO]{\thepage}	
}
\fancypagestyle{plain}{
	\fancyhf{}
	\fancyfoot[CE,CO]{\fontfamily{\rmdefault}\selectfont /@ variables.title @/}
	\renewcommand{\headrulewidth}{0pt}
	\fancyfoot[LE,RO]{\thepage}	
}
\pagestyle{dndstyle}

\newcommand{\inches}{\thinspace\ensuremath{{}^{\prime\prime}}}

\begin{document}
%\fontfamily{\familydefault}\selectfont
\frenchspacing

\begin{titlepage}
\hspace*{6.5in}.
\thispagestyle{empty}
/$ if variables.coverimage $/
\BgThispage
/$ endif $/

\begin{textblock*}{\paperwidth}(0in,10in)
\definecolor{antique_white}{HTML}{FAEBD7}
\begin{tcolorbox}[enhanced,
  arc=0mm,
  opacityback=0.5,
  opacityframe=0.5, 
  colback=antique_white,
  colframe=antique_white
]
\centerline{/@variables.code@/}\vspace*{6pt}
\centerline{\scshape\Huge /@variables.title@/}\vspace*{6pt}
\centerline{\textit{A/$if variables.playtime == '8'$/n/$endif$/ /@variables.playtime@/ hour adventure for characters levels /@intvars.tierlevels@/}}

\end{tcolorbox}
\end{textblock*}

\clearpage
\vfill
\end{titlepage}

\setcounter{tocdepth}{1}
\setcounter{secnumdepth}{0}

\begingroup
\let\onecolumn\twocolumn
\let\cleardoublepage\relax
\let\clearpage\relax
/$ if intvars.longtoc $/
\small
/$ endif $/
\tableofcontents
\endgroup
\newpage
/@ body @/
\vspace*{0.5in}
\noindent {\scshape\Large Credits}

/$ if variables.author $/ \noindent \textbf{Author}: /@variables.author@/ \par /$endif$/
/$ if variables.artists $/ \noindent \textbf{Art Work}: /@variables.artists@/ \par /$endif$/
/$ if variables.editors $/ \noindent \textbf{Editing Assistance}: /@variables.editors @/ \par /$endif$/
/$ if variables.playtesters $/\noindent \textbf{Playtesters}: /@ variables.playtesters @/ \par /$ endif $/
\noindent \textbf{Last Revised:} \today
\vspace*{0.25in}

/$ if splices.credits $/
\par /@splices.credits.body@/
\vspace*{0.25in}
/$ endif $/
\vfill
\noindent {\small DUNGEONS \& DRAGONS, D\&D, Wizards of the Coast, Forgotten Realms, the dragon ampersand,
and all other Wizards of the Coast product names, and their respective logos are trademarks of 
Wizards of the Coast in the USA and other countries. This work contains material that is copyright 
Wizards of the Coast and/or other authors. Such material is used with permission under the 
Community Content Agreement for Dungeon Masters Guild. All other original material in this 
work is copyright /@ variables.year @/ by /@ variables.author @/ and published under the 
Community Content Agreement for Dungeon Masters Guild.}

\chapter*{Preliminaries}

/$ if splices.prelimintro $/
/@splices.prelimintro.body@/
/$ else $/

/@variables.title@/ is an adventure for Dungeons and Dragons 5th edition, originally created for
/@variables.convention@/ /@variables.year@/. It is set in Wizards of the Coast's Forgotten Realms
and is intended and legal for play in the D\&D Adventurers League. For more information please 
visit the D\&D Adventurers League home at:\\ \url{http://www.dndadventurersleague.org}

/$ endif $/

\section*{Running the Adventure}

%There should not be fore play at the table. Adventurers League is a family friendly organization.

Before running this adventure, the Dungeon Master should read the entire adventure, 
including all the box text. Box text is often used to either provide important details
about the environment or advance the plot, and the information it contains is often not
repeated elsewhere in the text.

Combat encounters appear at the end of the chapter in which they take place. This allows them to be
laid out in a way that minimizes the amount of page flipping required to find stat blocks while running
the encounter. Because encounters may contain player box text and plot advancement, it may be useful to
flip ahead and read the encounters as they are referenced in the text. If you plan on using miniatures to
run the adventure, a list of the miniatures required is in appendix
A: \hyperref[appendix_adventure_summary]{Adventure Summary}.

\section*{Adventure Length}
This adventure is expected to take /@variables.hours@/ hours to run as written. This assumes
the group stays on topic, that the party is consistently moving forward at a reasonable pace, 
and that each turn of combat is resolved quickly. If the group has trouble staying on topic,
use your position as Dungeon Master to remind the group about the adventure at hand and keep
the story moving. When the game maintains forward momentum and moves forward quickly the group
will find it easier to focus.

A party that is not making consistent progress and is out of options will quickly become frustrated.
If the players get lost or off track, feel free to provide hints or clues to get them back
on track. If the characters are unable to overcome a particular obstacle or challenge they may
need to be presented with additional options for moving forward.

Combat should be tense and exciting. Once a particular battle has reached the point where the
outcome is no longer in doubt, consider calling the fight. Once the parties opponents have lost the
ability to meaningfully harm the party further action does not serve to advance the plot. Narrate the
ending and keep the adventure moving forward.
\vfill
\pagebreak
\section*{Adjusting the Adventure}

\textbf{This adventure is meant for a group of five to six characters, of levels /@intvars.tierlevels@/. It is 
balanced for a party of /@intvars.optlevels@/ level.} Parties of that level will find this 
adventure suitably challenging if they face the Normal version of the encounters. Larger parties,
parties with higher level characters, or
those that work especially well together may find the Strong Party option of the encounters more rewarding.
If the party is smaller, has a lower average level, or does not have a good mix of party roles the
Dungeon Master may need to use the Weak Party option to avoid overwhelming the characters.

Player skill (not character level) can make a big difference in how a party copes with challenges. Players
new to the game will find easier encounters less frustrating. Experienced players will be able to tackle
more difficult fights than their character level may indicate and will often enjoy the challenge.

/$ if splices.preliminaries $/
/@splices.preliminaries.body@/
/$ endif $/

/$ for chp in chapters $/
/@ chp| usetemplate('chapter') @/
/$ endfor $/

\appendix
\chapter{Adventure Summary}
\label{appendix_adventure_summary}

\section*{Figures Needed}

If you plan on using miniatures to run the combat encounters in this adventure, you will find it useful 
to have the following figures on hand. It may be possible to run the adventure with fewer figures of
some type, but you will not need more than is shown in the table so long as the party does not
attempt to take on two encounters at once.

\begin{center}
\begin{dndtable}{Figures Needed}{Xr}
\textbf{Figure} & \textbf{\# Required}\\
/$ for fig in figures $/
/@ fig.0 @/ & /@ fig.1 @/\\
/$ endfor $/
\end{dndtable}
\end{center}

\section*{Experience Tables}

The following table gives the experience for each encounter, by encounter difficulty. The final row
gives the experience earned by a party if they defeated every encounter at the same difficulty
level.

\begin{center}
\begin{dndtable}{Encounter Experience by Difficulty}{X/$ for x in xpdata.difficulties $/r/$ endfor $/}
\textbf{Encounter} /$ for x in xpdata.difficulties $/ & \textbf{/@x@/}/$ endfor $/ \\
/$ for enc in xpdata.by_encounter $/
/@ " & ".join(enc) @/ \\ 
/$ endfor $/
\textbf{/@ "} & \\textbf{".join(xpdata.totalxpbydiff)@/}
\end{dndtable}
\end{center}

/$ if not variables.nodivxp $/
If the party completed every encounter at the same difficulty level, the following table gives the
experience earned by each member of the party, based on party size.

\begin{center}
\begin{dndtable}{Experience by Party Size}{Xrrrrr}
 & \textbf{3} & \textbf{4} & \textbf{5} & \textbf{6} & \textbf{7}\\
/$ for x in xpdata.party_size $/
/@ " & ".join(x) @/ \\ 
/$ endfor $/
\end{dndtable}
\end{center}
/$endif$/

/$ if quests $/
If the characters accomplished any of the following, they each receive the listed addition experience points.

\begin{center}
\begin{dndtable}{XP For Accomplishments}{Xr}
\textbf{Accomplishment} & \textbf{XP}\\
/$ for q in quests $/
/@ q.0 @/ & /@ q.1 @/\\
/$ endfor $/
\textbf{Total XP} & \textbf{/@totalquestxp@/ XP Each}\\
\end{dndtable}
\end{center}
/$endif$/

Regardless of monsters defeated, objectives accomplished, and party size the
\textbf{minimum XP award is  /@variables.minxp@/} and the \textbf{maximum XP award is /@variables.maxxp@/}.

\section*{Treasure}

/$ if cash $/
\begin{center}
\begin{dndtable}{Treasure By Value}{Xr}
/$ for loot in cash $/
/@ loot.0 @/ & /@ loot.1 @/\\
/$ endfor $/
\textbf{Total Value} & \textbf{/@ totalcash @/}
\end{dndtable}
\end{center}

/$ if not variables.nodivgp $/
If the party earned, found, looted, or otherwise acquired all cash rewards in the adventure,
the table below gives the split of the treasure based on party size.

\begin{center}
\small
\begin{dndtable}{Treasure By Party Size}{Xrrrrr}
\textbf{Party Size} & 3 & 4 & 5 & 6 & 7\\
\textbf{Total GP} /$ for x in cashsplit $/ & /@x@/ /$ endfor $/\\
\end{dndtable}
\end{center}
/$ endif $/
/$ endif $/

/$ if magicitems $/
\section*{Magic Items}
/$ for mi in magicitems $/
\label{magicitem_/@mi.name@/}
\centerline{\textbf{/@mi.name@/} /$ if mi.available > 1$/ (/@mi.available@/ available) /$endif$/}
\centerline{\textit{/@mi.category@/ (/@mi.rarity@/)}} 
/@mi.body@/

/$if mi.attuneby $/ \textit{This item requires attunement by /@mi.attuneby@/.} 
/$elif mi.attunement $/ \textit{This item requires attunement.} /$endif$/
\vspace*{.1in}
/$ endfor $/
/$ endif $/

/$ if consumables or scrolls $/
\section*{Consumables}
/$ if scrolls $/
\label{magicitem_scrolls}
\centerline{\textbf{Spell Scrolls}}
A spell scroll contains the spell. If that spell is on your spell list, you can use an action to read
the scroll and cast the spell without having to provide any of the spell's components. Otherwise, the scroll is
unintelligible. If the spell is of a higher level than you can normally cast, make an ability check using your
spellcasting ability against a DC of 10 + the spell's level. On a failed check there is no effect.
Regardless of success or failure, once used the scroll crumbles to dust.

\begin{dndtable}{Scrolls Available}{>{\raggedright\arraybackslash}Xr}
\textbf{Spell} & \textbf{\# Available}\\
/$ for sc in scrolls $/
/@sc.scrolltag@/ & /@sc.variables.available@/\\
/$ endfor $/
\end{dndtable}
\vspace*{.1in}
/$endif $/

/$ for mi in consumables $/
\label{magicitem_/@mi.name@/}
\centerline{\textbf{/@mi.name@/} /$ if mi.available > 1$/ (/@mi.available@/ available) /$endif$/}
\centerline{\textit{/@mi.category@/ (/@mi.rarity@/)}} 
/@mi.body@/

\vspace*{.1in}
/$ endfor $/
/$ endif $/

/$ if splices.noteable_items or spellbooks or mundane $/
\section*{Notable Items}
/$ if mundane $/
\subsection{Mundane Items}
The following mundane items are present in the adventure, and may be kept or
distributed by the party if found. They have no meaningful cash value if sold.
 
\begin{dnditemize}
/$ for it in mundane $/
\item /@it@/
/$ endfor $/
\end{dnditemize}
/$endif $/

/$for book in spellbooks $/
/@ book | usetemplate('spellbook') @/
/$endfor$/

/$ if splices.noteable_items $/
/@splices.noteable_items.body@/
/$ endif $/
/$ endif $/

/$ if storyawards $/
\section*{Story Awards}

/$ for st in storyawards $/
\par /@st.body2@/
\vspace{10pt}

/$ endfor $/

/$ for st in storyawards $/
%\centerline{\textbf{ /@st.variables.code @/ /@st.name@/}}
\label{/@st.label@/}
\begin{storyaward}{/@st.variables.code@/}{/@st.name@/}
/@st.body@/
\end{storyaward}
\vspace{15pt}
/$ endfor $/
/$ endif $/
\vfill

/$ if npcs $/
\chapter{Important Non-Player Characters}
\label{appendix_important_nonplayer_characters}

/$ for n in npcs $/
\par\noindent\label{npc_/@n.label@/}\textbf{/@n.name@/} /$if n.pronunciation$/ (/@n.pronunciation@/) /$endif$/\\
\textit{/@n.race@/, /@n.gender@//$if n.title$/, /@n.title@//$endif$/}\\
/@n.description@/
/$if n.quote $/
\textbf{Quote:} /@n.quote@/
/$endif$/

/$ if n.stats $/
/@ n.sblock | usetemplate('monster') @/
/$ endif $/
\vspace*{3mm}
/$ endfor $/
/$ endif $/

/$ if extra_monsters $/
\chapter{Additional Creatures}
\label{appendix_additional_creatures}
/$ for sblock in extra_monsters $/
/@ sblock | usetemplate('monster') @/
/$ endfor $/
/$endif$/

/$ if spellbook $/
\chapter{Spell Reference}
\label{appendix_spell_reference}
/$ for sp in spellbook $/
/@ sp | usetemplate('spell') @/
\vspace*{16pt}
/$endfor $/
/$endif$/

/$ if appendix $/
/$ for app in appendix $/
\onecolumn
\titlespacing*{\chapter}{0pt}{-16pt}{0pt}
\chapter{/@app.name@/}
\label{appendix_/@app.label@/}
\Figure*[placement=!h,max width=\textwidth, max height=0.945\textheight]{/@app.image@/}
/$ endfor $/
\twocolumn
\titlespacing*{\chapter}{0pt}{0pt}{0pt}
/$ endif $/

/$ if bibliography $/
\chapter{Bibliography}
\label{appendix_bibliography}
\begin{flushleft}


/$ for bib in bibliography $/

\par /@bib.author@/. ``/@ bib.title @/''. /$ if bib.publisher $/ 
Published by /@bib.publisher@/, /@bib.year@/. /$ endif $/ %/$ if bib.editor $/Edited by /@bib.editor@/. /$ endif $/ 
Available from \textit{/@ bib.site @/}.\\ 
{\small \href{/@bib.link@/}{/@bib.url@/}}
\vspace*{3mm}
/$ endfor $/
\end{flushleft}
/$ endif $/

\newpage
\centerline{}

\end{document}